@c charter.tex -- TeXinfo macros to set the default Roman font to Charter
@c
@def@charterversion{2012-03-05.01}
@c
@c--------------------------------------------------------------------------------
@c This file is edited by Andres Raba combining and borrowing from three files: 
@c palatino.tex from GNU Press,
@c texinfo.tex from Free Software Foundation, and
@c font_utopia.tex from font-change CTAN package created by Amit Raj Dhawan.
@c 
@c You can get original palatino.tex from GNU Press CVS:
@c cvs -z3 -d:pserver:anonymous@cvs.savannah.gnu.org:/sources/gnupress co gnupress
@c The procedure is documented by Sandip Bhattacharya at:
@c http://blog.sandipb.net/2009/02/12/getting-more-printable-pdfs-from-texinfo-manuals/
@c
@c The latest texinfo.tex is available from:
@c http://www.gnu.org/software/texinfo/
@c--------------------------------------------------------------------------------
@c
@c Copyright (C) 2003  Free Software Foundation, Inc.
@c
@c This charter.tex file is free software; you can redistribute it and/or
@c modify it under the terms of the GNU General Public License as
@c published by the Free Software Foundation; either version 2, or (at
@c your option) any later version.
@c
@c This charter.tex file is distributed in the hope that it will be
@c useful, but WITHOUT ANY WARRANTY; without even the implied warranty
@c of MERCHANTABILITY or FITNESS FOR A PARTICULAR PURPOSE.  See the GNU
@c General Public License for more details.
@c
@c You should have received a copy of the GNU General Public License
@c along with this charter.tex file; see the file LICENSE.  If not, write
@c to the Free Software Foundation, Inc., 59 Temple Place - Suite 330,
@c Boston, MA 02111-1307, USA.
@c
@c To use this module, go to your root .texi file and make sure
@c it reads like this:
@c     \input texinfo  @c -*-texinfo-*-
@c     @input charter

@message{Loading charter [version @charterversion]:}

@c Turn on the normal TeX command characters.
@catcode`\=0
\catcode`\%=14
% Set the font macro #1 to the font named #2#3. #4 is the point size.
% We save \setfont as \setfontorig, so we can restore it at the end of this
% section.
\catcode`\#=6
\let\setfontorig=\setfont
\def\setfont#1#2#3#4{\font#1=#2#3 at #4}
\catcode`\#=\other

% Set Charter as the default roman font face
\def\rmfontprefix{mdbch}

% Only define roman font attributes here.
\def\rmshape{r7t}
\def\rmbshape{b7t}
\def\bfshape{b7t}
\def\bxshape{b7t}
\def\itshape{ri7t}
\def\itbshape{bi7t}
\def\slshape{ro7t}
\def\slbshape{bo7t}
\def\scshape{rfc8t}
\def\scbshape{bfc8t}

% Set Bera Sans Mono as the default typewriter font face
\def\ttfontprefix{fvm}

% Define tt font attributes here for fvm
\def\ttshape{r8t}
\def\ttbshape{b8t}
\def\ttslshape{ro8t}
\def\ttbslshape{bo8t}

% Set MathDesign as the math italic and math symbols font face
\def\mtfontprefix{md}

% Define math font attributes here
\def\mtitshape{bchri7m}			% math italic
\def\mtsyshape{-chr7y}			% math symbol
\font\elevenex=  mdbchr7v at 17.4pt 	% math extension

\ifx\bigger\relax
  % not really supported.
  \def\textfontsize{17.4pt}
  \setfont\textrm\rmfontprefix\rmshape{\textfontsize}
\else
  % was 10pt:
  \def\textfontsize{16pt}
  \textleading = 20pt 
  \setfont\textrm\rmfontprefix\rmshape{\textfontsize}
\fi
\setfont\textbf\rmfontprefix\bfshape{\textfontsize}
\setfont\textit\rmfontprefix\itshape{16.4pt}		% Italic is smaller than roman, needs scaling
\setfont\textsl\rmfontprefix\slshape{\textfontsize}
\setfont\textsc\rmfontprefix\scshape{\textfontsize}
% For tt shapes:
\setfont\texttt\ttfontprefix\ttshape{14.0pt}		% Bera Sans Mono is bigger than Charter,
\setfont\textttsl\ttfontprefix\ttslshape{14.0pt}	% needs scaling
% For math:
\setfont\texti\mtfontprefix\mtitshape{16.4pt}		% MathDesign is smaller than Charter, 
\setfont\textsy\mtfontprefix\mtsyshape{16.4pt}		% needs scaling

% A few fonts for \defun, etc.
\setfont\defbf\rmfontprefix\bxshape{14.5pt} %was 1314
\def\df{\let\tentt=\deftt \let\tenbf = \defbf \bf}

% Fonts for indices, footnotes, small examples (9pt).
\def\smallfontsize{14pt}		% was 9pt
\setfont\smallrm\rmfontprefix\rmshape{\smallfontsize}
\setfont\smallbf\rmfontprefix\bfshape{\smallfontsize}
\setfont\smallit\rmfontprefix\itshape{14.35pt}
\setfont\smallsl\rmfontprefix\slshape{\smallfontsize}
\setfont\smallsc\rmfontprefix\scshape{\smallfontsize}
% For tt shapes:
\setfont\smalltt\ttfontprefix\ttshape{12.25pt}
\setfont\smallttsl\ttfontprefix\ttslshape{12.25pt}
% For math:
\setfont\smalli\mtfontprefix\mtitshape{14.35pt}
\setfont\smallsy\mtfontprefix\mtsyshape{14.35pt}

% Fonts for small examples (8pt).
\def\smallerfontsize{12.3pt}		% was 8pt
\setfont\smallerrm\rmfontprefix\rmshape{\smallerfontsize}
\setfont\smallerbf\rmfontprefix\bfshape{\smallerfontsize}
\setfont\smallerit\rmfontprefix\itshape{12.6pt}
\setfont\smallersl\rmfontprefix\slshape{\smallerfontsize}
\setfont\smallersc\rmfontprefix\scshape{\smallerfontsize}
% For tt shapes:
\setfont\smallertt\ttfontprefix\ttshape{10.76pt}
\setfont\smallerttsl\ttfontprefix\ttslshape{10.76pt}
% For math:
\setfont\smalleri\mtfontprefix\mtitshape{12.6pt}
\setfont\smallersy\mtfontprefix\mtsyshape{12.6pt}

% Fonts for title page:
\def\titlefontsize{42.7pt} 
\setfont\titlerm\rmfontprefix\rmbshape{\titlefontsize}
\setfont\titleit\rmfontprefix\itbshape{\titlefontsize}
\setfont\titlesl\rmfontprefix\slbshape{\titlefontsize}
\let\titlebf=\titlerm
\setfont\titlesc\rmfontprefix\scbshape{\titlefontsize}
\setfont\titlei\mtfontprefix\mtitshape{\titlefontsize}
\setfont\titlesy\mtfontprefix\mtsyshape{\titlefontsize}
\def\authorrm{\secrm}
\def\authortt{\sectt}

% Chapter (and unnumbered) fonts (17.28pt).
\def\chapfontsize{25pt}
\setfont\chaprm\rmfontprefix\rmbshape{\chapfontsize}
\setfont\chapit\rmfontprefix\itbshape{\chapfontsize}
\setfont\chapsl\rmfontprefix\slbshape{\chapfontsize}
\let\chapbf=\chaprm
\setfont\chapsc\rmfontprefix\scbshape{\chapfontsize}
% For tt shapes:
\setfont\chaptt\ttfontprefix\ttbshape{22.1pt}
\setfont\chapttsl\ttfontprefix\ttbslshape{22.1pt}
% For math:
\setfont\chapi\mtfontprefix\mtitshape{\chapfontsize}
\setfont\chapsy\mtfontprefix\mtsyshape{\chapfontsize}

% Section fonts (14.4pt).
\def\secfontsize{21pt}
\setfont\secrm\rmfontprefix\rmbshape{\secfontsize}
\setfont\secit\rmfontprefix\itbshape{\secfontsize}
\setfont\secsl\rmfontprefix\slbshape{\secfontsize}
\let\secbf\secrm
\setfont\secsc\rmfontprefix\scbshape{\secfontsize}
% For tt shapes:
\setfont\sectt\ttfontprefix\ttbshape{18.6pt}
\setfont\secttsl\ttfontprefix\ttbslshape{18.6pt}
% For math:
\setfont\seci\mtfontprefix\mtitshape{\secfontsize}
\setfont\secsy\mtfontprefix\mtsyshape{\secfontsize}

% Subsection fonts (13.15pt).
\def\ssecfontsize{19pt}
\def\ttssecfontsize{17.1pt}
\setfont\ssecrm\rmfontprefix\rmbshape{\ssecfontsize}
\setfont\ssecit\rmfontprefix\itbshape{\ssecfontsize}
\setfont\ssecsl\rmfontprefix\slbshape{\ssecfontsize}
\let\ssecbf\ssecrm
\setfont\ssecsc\rmfontprefix\scbshape{\ssecfontsize}
% For tt shapes:
\setfont\ssectt\ttfontprefix\ttbshape{16.8pt}
\setfont\ssecttsl\ttfontprefix\ttbslshape{16.8pt}
% For math:
\setfont\sseci\mtfontprefix\mtitshape{\ssecfontsize}
\setfont\ssecsy\mtfontprefix\mtsyshape{\ssecfontsize}

% Reduced fonts for @acro in text (10pt).
\def\reducedfontsize{14.5pt}
\setfont\reducedrm\rmfontprefix\rmshape{\reducedfontsize}
\setfont\reducedbf\rmfontprefix\rmbshape{\reducedfontsize}
\setfont\reducedit\rmfontprefix\itshape{\reducedfontsize}
\setfont\reducedsl\rmfontprefix\slshape{\reducedfontsize}
\setfont\reducedsc\rmfontprefix\scshape{\reducedfontsize}
% For tt shapes:
\setfont\reducedtt\ttfontprefix\ttshape{12.8pt}
\setfont\reducedttsl\ttfontprefix\ttslshape{12.8pt}
% For math:
\setfont\reducedi\mtfontprefix\mtitshape{\reducedfontsize}
\setfont\reducedsy\mtfontprefix\mtsyshape{\reducedfontsize}

% The smallcaps and symbol fonts should actually be scaled \magstep1.5,
% but that is not a standard magnification.

% Fonts for short table of contents.
\setfont\shortcontrm\rmfontprefix\rmshape{17.4pt}
\setfont\shortcontbf\rmfontprefix\bxshape{17.4pt}
\setfont\shortcontsl\rmfontprefix\slshape{17.4pt}

% Set keyfont as well.
\setfont\keyrm\rmfontprefix\rmshape{11.6pt}
\setfont\keysy\mtfontprefix\mtsyshape{13.1pt}

% Fonts for ordinal number superscripts 
% (1st, 2nd, 6th, etc.) in text and footnote
\setfont\ordrm\rmfontprefix\rmshape{11pt}
\setfont\fordrm\rmfontprefix\rmshape{9.6pt}

\let\subtitlerm=\textbf
\def\subtitlefont{\subtitlerm \normalbaselineskip = 24pt \normalbaselines}

% Adapted from font_utopia.tex (original author Amit Raj Dhawan):

\font\eightrm=	mdbchr7t at 12.3pt
\font\sixrm=	mdbchr7t at 8.7pt

\font\eighti=	mdbchri7m at 12.3pt
\font\sixi=	mdbchri7m at 8.7pt

\font\eightsy=	md-chr7y at 12.3pt
\font\sixsy=	md-chr7y at 8.7pt

\font\eightex=	mdbchr7v at 12.3pt
\font\sixex=	mdbchr7v at 8.7pt

\font\eightit=	mdbchri7t at 12.3pt
\font\sixit=	mdbchri7t at 8.7pt

\font\eightsl=	mdbchro7t at 12.3pt
\font\sixsl=	mdbchro7t at 8.7pt

\font\eightbf=	mdbchb7t at 12.3pt
\font\sixbf=	mdbchb7t at 8.7pt

\font\eighttt=	fvmr8t at 12.3pt
\font\sixtt=	fvmr8t at 8.7pt

% Family 0 (roman text)
\scriptfont0=\eightrm
\scriptscriptfont0=\sixrm
%
% Family 1 (math italic)
\scriptfont1=\eighti
\scriptscriptfont1=\sixi
%
% Family 2 (math symbol)
\scriptfont2=\eightsy
\scriptscriptfont2=\sixsy
%
% Family 3 (math extension)
\textfont3=\elevenex
\scriptfont3=\eightex
\scriptscriptfont3=\sixex
%
% Family 4 (italic text)
%\def\it{\fam=\itfam \tenit}%
\scriptfont\itfam=\eightit
\scriptscriptfont\itfam=\sixit
%
% Family 5 (slanted text)
\scriptfont\slfam=\eightsl
\scriptscriptfont\slfam=\sixsl
%
% Family 6 (boldface text)
\scriptfont\bffam=\eightbf
\scriptscriptfont\bffam=\sixbf
%
% Family 7 (typewriter text)
\scriptfont\ttfam=\eighttt
\scriptscriptfont\ttfam=\sixtt


\let\setfont=\setfontorig
\def\setfontorig{\relax}

% Restore the TeXinfo character set.
\catcode`\\=\active
@catcode`@%=@other

@c Set initial fonts (again)
@textfonts
@rm

@c Local variables:
@c eval: (add-hook 'write-file-hooks 'time-stamp)
@c page-delimiter: "^\\\\message"
@c time-stamp-start: "def\\\\charterversion{"
@c time-stamp-format: "%:y-%02m-%02d.%02H"
@c time-stamp-end: "}"
@c End:
